\begin{table}[htbp]
\centering
\begin{tabular}{l l l l l l l l}
\toprule
\textbf{Nivel Meta (Protocolo Cognitivo)} & \textbf{docs/manual-operativo.md} & \textbf{Configurar el comportamiento del modelo, definir restricciones operativas y estándares de respuesta antes de procesar código.} & \textbf{Al inicio de la sesión o dentro del System Prompt.} & \textbf{Brevedad en tokens, arquitectura de "Lazy Loading" y uso de orquestación para tareas extensas.} & \textbf{Protocolo de recuperación, modelo de orquestador/sub-agentes y ancla de la verdad contra alucinaciones.} & \textbf{Leer el manual para entender las restricciones y el estándar de respuesta ante instrucciones iniciales.} & \textbf{[1, 2]} \\
\midrule
Nivel Contexto (Mapa del Territorio) & Agent.md & Situar al agente en el dominio específico del proyecto, proporcionando la arquitectura, estándares y mapa del repositorio. & Inmediatamente después del manual operativo, como contexto principal. & Máximo 500 líneas (idealmente entre 250 y 500) para mantener una alta densidad de señal/ruido. & Visión general, mapa del repositorio, índice de skills, flujos de trabajo y stack tecnológico. & Analizar para comprender la estructura del repositorio, objetivos actuales y ubicación de recursos. & [1-3] \\
Nivel Funcional (Activación bajo demanda) & skills/*.md & Cargar capacidades técnicas específicas de forma modular para optimizar el uso de la ventana de contexto. & Solo cuando se activa un disparador específico durante la ejecución de la tarea; nunca de forma proactiva. & Aislamiento de capacidades; si una skill supera las 500 líneas, debe refactorizarse y dividirse. & Trigger (activador), Scope (alcance), Tools (herramientas) y Resources (plantillas One-Shot). & Autoinvocación: buscar y leer la skill correspondiente cuando la solicitud del usuario coincide con el trigger. & [1-3] \\
\bottomrule
\end{tabular}
\end{table}

\section*{引用来源}
\begin{itemize}
\item [1] guia-orden-instrucciones.pdf
\item [2] manual-operativo.md
\item [3] Agent.md
\end{itemize}